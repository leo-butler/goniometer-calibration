\documentclass[a4paper]{amsart}
%% packages
\usepackage{showkeys,color,ifthen,comment}

\newcommand{\estimate}[1]{\hat{#1}}
\newcommand{\R}[0]{{\mathbf R}}
\newcommand{\set}[1]{\left\{#1\right\}}
\newcommand{\st}[0]{\,:\,}
\newcommand{\ip}[2]{\langle #1,#2 \rangle}
\newcommand{\norm}[1]{|#1|}
\newcommand{\dist}[2]{{\mathrm d}(#1,#2)}
\newcommand{\powerset}[2]{#1^{(#2)}}
\newcommand{\argmin}[2]{{\displaystyle\mathrm{argmin}_{#1}\,#2}}
\newcommand{\textem}[1]{{\em #1}}

%% a customised question/comment environment
%% for use in comments.tex
\definecolor{gray}{rgb}{0.6,.6,0.6}
\definecolor{green}{rgb}{0,1,0}
\newcommand{\LB}{\item[\textbf{LB:}]}
\newcommand{\RG}{\item[\textbf{RG:}]}
\newenvironment{qc}[1][6mm]{
\def\lmiqc{\leftmargini}
\setlength{\leftmargini}{#1}
\begin{enumerate}}{
\end{enumerate}
\setlength{\leftmargini}{\lmiqc}}
\newenvironment{qcitem}[1][0]{\item\begin{enumerate}
    \ifthenelse{\equal{#1}{done}}{\color{gray}}{\ifthenelse{\equal{#1}{new}}{\color{red}}{}}
    }{\end{enumerate}}

%% a customised comment environment for the paper

%% in text:
\newcommand{\inclcomments}[0]{1}
\newcommand{\includecomments}[1]{%
  \ifthenelse{\equal{#1}{}}{%
    \inclcomments{}%
  }{%
    \renewcommand{\inclcomments}[0]{#1}}%
}

\newsavebox{\acommentbox}
\newenvironment{acomment}[1][{false}]{%
  \def\includethiscomment{#1}
  \begin{lrbox}{\acommentbox}%
    \raggedright\ttfamily\footnotesize\color{red}}{%
  \end{lrbox}%
  \ifthenelse{\equal{{\inclcomments}}{{false}}\or\equal{{\includethiscomment}}{{done}}}{%
    }{\usebox{\acommentbox}}
}

%% marginal:
\setlength{\marginparwidth}{35mm}
\newsavebox{\mcommentbox}
\newenvironment{mcomment}[1][{false}]{%
  \def\includethiscomment{#1}
  \begin{lrbox}{\mcommentbox}%
    \begin{minipage}{\marginparwidth}%
      \raggedright\ttfamily\footnotesize\color{red}}{%
    \end{minipage}%
  \end{lrbox}%
  \marginpar{%
    \ifthenelse{\equal{{\inclcomments}}{{false}}\or\equal{{\includethiscomment}}{{done}}}{%
      }{\usebox{\mcommentbox}}
  }%
}
\newcommand{\commentm}[2][0]{\begin{mcomment}[{#1}]#2\end{mcomment}}

%%% Local Variables: 
%%% mode: latex
%%% TeX-master: "gc"
%%% End: 

%% end of definitions.tex

\includecomments{1}

\begin{document}
\section{Preamble}
\label{sec:pre}
The purpose of this file is to communicate ideas/comments on the
paper. Feel free to delete comments/questions that you have raised if
you feel they have been resolved. At the end, this section should be
the only non-empty section.

Notation: vXXlYYcZZ will refer to version XX, line YY and optionally
column (or character) ZZ on line YY.

Document structure: it should be structured like
\begin{verbatim}
\begin{qc}        <-- begin the list of questions/comments
  \begin{qcitem}  <-- begin a question/comment
    \LB           <-- comments by Leo
    \RG           <-- comments by Roman
  \end{qcitem}    <-- end a question/comment
  .
  .               <-- more qcitem environments
  .
  \begin{qcitem}[done]  <-- a finished question/comment (appears in gray)
    .
    .
  \end{qcitem}          <-- end of same question/comment
\end{qc}          <-- end list of questions/comments
\end{verbatim}
so there is need for only one \verb#qc# environment, and distinct
questions/comments can be put into distinct \verb#qcitem#
environments. There is no need to use {\it italics} to show who said
what or to use \verb#\\# to end a line, just start your comment on a
new line and preface your comment with \verb#\RG#.

\section{Comments \& Questions}
\label{sec:cq}
\begin{qc}
  \begin{qcitem}
    \LB Please take a look at this file.
  \end{qcitem}
\end{qc}

\bibliographystyle{amsplain}
\bibliography{gc}
\end{document}

%%% Local Variables: 
%%% mode: latex
%%% TeX-region: "Comments"
%%% TeX-master: t
%%% eval: (load-file "customisation.el")
%%% End: 
