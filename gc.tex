%% -*- Mode: LaTeX -*-
%%
%% $Id:$
%%
%% Author: Leo Butler (l.butler@ed.ac.uk)
%%
%% This file is LaTeX code (http://www.tug.org/)
%%
%% It is free software; you can redistribute it and/or modify
%% it under the terms of the GNU General Public License as published by
%% the Free Software Foundation; either version 3 of the License, or (at your
%% option) any later version.
%%
%% This software is distributed in the hope that it will be useful, but
%% WITHOUT ANY WARRANTY; without even the implied warranty of MERCHANTABILITY
%% or FITNESS FOR A PARTICULAR PURPOSE. See the GNU General Public
%% License for more details.
%%
%% You should have received a copy of the GNU General Public License
%% along with this file. If not, see http://www.gnu.org/licenses/.
%%
\documentclass[a4paper,reqno]{amsart}
%% packages
\usepackage{showkeys,color,ifthen,comment}

\newcommand{\estimate}[1]{\hat{#1}}
\newcommand{\R}[0]{{\mathbf R}}
\newcommand{\set}[1]{\left\{#1\right\}}
\newcommand{\st}[0]{\,:\,}
\newcommand{\ip}[2]{\langle #1,#2 \rangle}
\newcommand{\norm}[1]{|#1|}
\newcommand{\dist}[2]{{\mathrm d}(#1,#2)}
\newcommand{\powerset}[2]{#1^{(#2)}}
\newcommand{\argmin}[2]{{\displaystyle\mathrm{argmin}_{#1}\,#2}}
\newcommand{\textem}[1]{{\em #1}}

%% a customised question/comment environment
%% for use in comments.tex
\definecolor{gray}{rgb}{0.6,.6,0.6}
\definecolor{green}{rgb}{0,1,0}
\newcommand{\LB}{\item[\textbf{LB:}]}
\newcommand{\RG}{\item[\textbf{RG:}]}
\newenvironment{qc}[1][6mm]{
\def\lmiqc{\leftmargini}
\setlength{\leftmargini}{#1}
\begin{enumerate}}{
\end{enumerate}
\setlength{\leftmargini}{\lmiqc}}
\newenvironment{qcitem}[1][0]{\item\begin{enumerate}
    \ifthenelse{\equal{#1}{done}}{\color{gray}}{\ifthenelse{\equal{#1}{new}}{\color{red}}{}}
    }{\end{enumerate}}

%% a customised comment environment for the paper

%% in text:
\newcommand{\inclcomments}[0]{1}
\newcommand{\includecomments}[1]{%
  \ifthenelse{\equal{#1}{}}{%
    \inclcomments{}%
  }{%
    \renewcommand{\inclcomments}[0]{#1}}%
}

\newsavebox{\acommentbox}
\newenvironment{acomment}[1][{false}]{%
  \def\includethiscomment{#1}
  \begin{lrbox}{\acommentbox}%
    \raggedright\ttfamily\footnotesize\color{red}}{%
  \end{lrbox}%
  \ifthenelse{\equal{{\inclcomments}}{{false}}\or\equal{{\includethiscomment}}{{done}}}{%
    }{\usebox{\acommentbox}}
}

%% marginal:
\setlength{\marginparwidth}{35mm}
\newsavebox{\mcommentbox}
\newenvironment{mcomment}[1][{false}]{%
  \def\includethiscomment{#1}
  \begin{lrbox}{\mcommentbox}%
    \begin{minipage}{\marginparwidth}%
      \raggedright\ttfamily\footnotesize\color{red}}{%
    \end{minipage}%
  \end{lrbox}%
  \marginpar{%
    \ifthenelse{\equal{{\inclcomments}}{{false}}\or\equal{{\includethiscomment}}{{done}}}{%
      }{\usebox{\mcommentbox}}
  }%
}
\newcommand{\commentm}[2][0]{\begin{mcomment}[{#1}]#2\end{mcomment}}

%%% Local Variables: 
%%% mode: latex
%%% TeX-master: "gc"
%%% End: 

%% end of definitions.tex

\title[Goniometer]{Calibration of a Goniometer with Constrained Least
  Squares}
\author{Leo T. Butler \& Roman Grothausmann}
\email{l.butler@ed.ac.uk}

\begin{document}
\begin{abstract}
  This note shows how a goniometer may be calibrated using the method
  of constrained least squares. We obtain error-bounds also.
\end{abstract}
\maketitle

\section{Introduction}
\label{sec:intro}

This note applies the geometric technique of constrained least
squares, or more generally, constrained optimization to the problem of
calibrating a physical instrument.

\subsection{The Goniometer and its Calibration}
\label{sec:goniometer-intro}

\subsection{The Geometric Model}
\label{sec:geometric-model}

From the description of the calibration problem, we have a number of
planes $P_i$ which intersect pairwise in the same line $\ell$. Our
measurements are of points $y_{ij}=x_{ij}+\epsilon_{ij}$, where
$x_{ij}$ lies on the plane $P_i$ and $\epsilon_{ij}$ is an error in
the measurement. We want to develop an estimator $\estimate{\ell}$ of
$\ell$ from the measurements $y_{ij}$.

To explain how to do this, a few observations are in order. First,
each {\em oriented} line $\ell$ in $\R^3$ is uniquely determined by
the point $p \in \ell$ that is closest to $0$ and its unit direction
vector $v$; an {\em unoriented} line is uniquely determined by $p$ and
$\set{\pm v}$. That is,
\begin{align*}
  L_+ &= \set{ (p,v) \in \R^3 \times \R^3 \st \ip{p}{v}=0,\norm{v}=1 }
    && =\textrm{set of oriented lines in }\R^3,\\
    L &= L_+/\sim, && =\textrm{set of unoriented lines in }\R^3,
\end{align*}
where $(p,v) \sim (p,-v)$. The set $L_+$ of oriented lines is the
tangent bundle of the unit sphere in $\R^3$; the set $L$ of unoriented
lines does not have such a nice description.%%%%%%%%%%%%%%%%%%%%%%%%%
\footnote{It is not the tangent bundle of real projective plane;
  rather, it is the pull-back bundle of the tangent bundle of $S^2$
  under the projection map from $S^2$ to the real projective plane.}
\ %%%%%%%%%%%%%%%%%%%%%%%%%%%%%%%%%%%%%%%%%%%%%%%%%%%%%%%%%%%%%%%%%%%
One can measure distance between `points' (i.e. oriented lines) $\ell_+=(p,v)$
and $\ell'_+=(p',v')$ in $L_+$ by the euclidean distance on $\R^3
\times \R^2$:
\begin{equation}
  \label{eq:distance-between-lines}
  \dist{\ell_+}{\ell'_+}^2 = \norm{p-p'}^2 + \norm{v-v'}^2
\end{equation}
and the corresponding distance between two unoriented lines is defined
by
\begin{equation}
  \label{eq:distance-between-unoriented-lines}
  \dist{\ell}{\ell'}^2 = \norm{p-p'}^2 + \min\set{ \norm{v-v'}^2,\norm{v+v'}^2}
\end{equation}

Let $Q_i=\set{y_{ij}=x_{ij}+\epsilon_{ij} \st j=1,\ldots,n_i}$ be a
set of non-collinear points, where $x_{ij} \in P_i$ and
$\epsilon_{ij}$ is a measurement error. Each set of $3$ points in
$Q_i$ defines a plane that is approximately $P_i$. Let
$\powerset{Q_i}{3}$ be the set of planes defined by each $3$-subset of
$Q_i$. If one supposes that, in the absence of noise in the data,
there is a unique line $\ell$ such that for all $i\neq j$, $P_i \cap
P_j = \ell$, then almost surely each pair
$\Pi_i\in\powerset{Q_i}{3}$, $\Pi_j\in\powerset{Q_j}{3}$ intersects in
a unique line that is approximately $\ell$:
  \begin{align}
    \label{eq:plane-intersection}
    \Pi_i \cap \Pi_j &= \tilde{\ell}.
  \end{align}
Thus, the function
\begin{align}
  \label{eq:objective-function}
  \omega(\ell) := \sum_{i < j}
  \sum_{\Pi_i\in\powerset{Q_i}{3},\Pi_j\in\powerset{Q_j}{3}} \dist{\Pi_i \cap \Pi_j}{\ell}^2
\end{align}
is a natural choice of objective function. Our estimator of the line
$\ell$ is defined by
\begin{align}
  \label{eq:line-estimator}
  \estimate{\ell} := \argmin{\ell}{\omega}.
\end{align}

The distribution of the estimator $\estimate{\ell}$ is approximated by
Monte Carlo methods, which allow us to estimate the error in
$\estimate{\ell}$.

\subsubsection{Why the planes are not estimated}
\label{sec:why-no-planes}

Let us remark that we could estimate each plane $P_i$ individually,
since the set of (unoriented) planes in $\R^3$ is parameterised by
\begin{align}
  \label{eq:planes-in-r3}
  \Pi_+ &= \set{ (n,c) \st n\in\R^3,\norm{n}=1, c\in\R }
  &&=\textrm{oriented planes}\\
  \Pi &= \Pi_+/\sim
  &&=\textrm{unoriented planes}
\end{align}
where $(n,c) \sim (-n,-c)$. Thus, the distance between two planes
$P_+=(n,c)$, $P'_+=(n',c')$ is
\begin{align}
  \label{eq:dplane}
  \dist{P_+}{P'_+}^2 &= \norm{n-n'}^2 + \norm{c-c'}^2\\
\intertext{and the distance between unoriented planes is}
  \dist{P}{P'}^2 &= \min\set{ \norm{n-n'}^2 + \norm{c-c'}^2,  \norm{n+n'}^2
    + \norm{c+c'}^2 }.
\end{align}
One can define an objective function for each $i$
\begin{align}
  \label{eq:plane-objectivefn}
  \Omega_i(P) &:= \sum_{\Pi_i \in \powerset{Q_i}{3}} \dist{\Pi_i}{P}^2
\end{align}
and obtain $\estimate{P}_i=\argmin{P}{\Omega_i}$.

There are two distinct problems with this approach. First, almost
surely, the lines $\estimate{P}_i \cap \estimate{P}_j$ are distinct
for distinct $i \neq j$. This means that even with the estimated
planes, one must go on and estimate the best fit line. Second, and
more fundamental, the transformation from pairs of planes to their
intersection, a line, is non-linear. This means that even if all the
estimated planes intersected in a single line, this line would be a
biased estimator. The issue here is no different from the reason why,
if one has mean zero variates $x_i$ and one wants to estimate the
population \textem{variance}, one does first estimate the sample mean
and computes its square.


\bibliographystyle{amsplain}
\bibliography{gc}
\end{document}



%%% Local Variables: 
%%% mode: latex
%%% TeX-master: t
%%% eval: (load-file "customisation.el")
%%% End: 

% end of gc.tex 
